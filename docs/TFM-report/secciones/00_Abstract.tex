
%%--------------
\newpage
%%--------------

\chapter*{Abstract}


This thesis focuses on the specification and development of link discovery methods, as well as the implementation of a knowledge graph linker. In other words, it is a Resource Description Framework (RDF) linking task that produces connections between RDF resources from different or the same dataset, like DBpedia, Wikidata, etc. 

RDF linking is based on one or more link rules that specify the conditions that two RDF resources can be considered the same, such as text similarity or geographical position. SPARQL Protocol and RDF Query Language (SPARQL) is the most widely used and standard query language and protocol for Linked Open Data on the web or RDF triplestores. So, we built a Sparql-based application to connect the knowledge graphs, and the Sparql engine we chose was Apache Jena, a free and open source Java framework for building Semantic Web and Linked Data applications. We implemented several linking rules and evaluated the results with multiple datasets from the Ontology Alignment Evaluation Initiative (OAEI) competition in 2021, and we achieved excellent results.





\hspace{10pt}
%TC:ignore
\keywords{RDF, SPARQL, Linking algorithm, Apache Jena, OAEI}
%%%%%%%%%%%%%%%%%%%%%%%%%%%%%%%%%%%%%%%%%%%%%%%%%%%%%%%%%%%
%% End of the abstract.
%%%%%%%%%%%%%%%%%%%%%%%%%%%%%%%%%%%%%%%%%%%%%%%%%%%%%%%%%%%
